Reviewing the literature around multi-parent genetic operators highlighted two operators: fitness-based scanning and diagonal crossover. These two strategies have shown success as multi-parent genetic operators, and are the conceptual descendants of traditional genetic operators.

\subsection*{Fitness-Based Scanning}
The first operator from the literature that has been implemented and tested is fitness-based scanning as proposed by Eiben, et al\cite{Eiben94}. The generalized form of scanning iterates through the empty genetic sequence of a child and determines its value based upon the values present in $n$ selected parents\cite{Eiben91}. Fitness-based scanning makes a roulette wheel selection to choose each allele. This procedure is further demonstrated in Figure-\ref{FBScan} At each allele, the probability that a parent, from the population $S$, named $i$ with a fitness value of $f(i)$, will donate its allele to the child's genetic sequence has probability $P(i)$ as described below\cite{Eiben94}: 

\[ P(i) = \frac{f(i)}{\sum\limits_{i \in S} f(i)} \]

\noindent Thus, the expected number of alleles inherited from a parent $i$ is $E(i)$\cite{Eiben94}:

\[ E(i) = P(i) *(\text{ Chromosome length }) \]


%
% FB-Scan Code
%
\begin{algorithm}
	\SetAlgoLined
	\KwData{ \emph{Parents}, a set of $n$ parent genetic sequences} 

\SetKwData{Parents}{Parents} \SetKwData{Value}{value} \SetKwData{aye}{i} \SetKwData{jay}{j} \SetKwData{pee}{p} \SetKwData{child}{child} \SetKwData{select}{select} \SetKwData{wheel}{wheel} \SetKwData{TotalFitness}{TotalFitness}\SetKwData{size}{size}\SetKwData{SP}{SelectedParent}
	
\KwResult{A new genetic sequence \emph{child}}
	\BlankLine

	\tcp{Calculate the \TotalFitness of \Parents}
	\For{$\aye \gets 0$ \KwTo $\Parents.\size$}{

		$\TotalFitness\gets \TotalFitness + \Parents[\aye].getFitnessValue()$
	}

	\BlankLine
	\tcp{Build the roulette wheel}
	$\wheel[0] \gets \Parents[0].getFitnessValue() / \TotalFitness$ \\
	\For{$\aye \gets 1$ \KwTo $\Parents.\size$}{

		$\wheel[\aye] \gets \wheel[\aye - 1] + \Parents[\aye].getFitnessValue() / \TotalFitness$
	}

	\BlankLine
	\tcp{Build the new genetic sequence}
	\For{$\aye \gets 0$ \KwTo $\child.length$}{

		$\select \gets $Random(0,1) \\
		
		\BlankLine
		\tcp{Find which parent we selected}
		\For{$\jay \gets 0$ \KwTo $\Parents.\size$}{

			\BlankLine
			\If{$\wheel[\jay] >= \select$}{
     	                     	$\SP \gets \Parents[\jay]$ \\
				break
			}
		}	
		$\child[\aye] \gets \SP[\aye]$

	}
				

	\caption{Fitness Based Scanning Pseudocode}
	\label{FBScan}
\end{algorithm}

\subsection*{Diagonal Crossover}
Traditional crossover takes two genetic sequences, chooses a splitting point, and swaps the portions after that point to create two new genetic sequences. \emph{Diagonal crossover} was introduced by Eiben, et al. to extend the concept of crossover into the realm of multi-parent genetic operators\cite{Eiben03}. A diagonal crossover of arity $n$ can be described easily. Take $n$ individuals from a population, select $n-1$ crossover points, and create $n$ children by selecting one segment from each piece of the genetic sequences given\cite{Eiben95}. Figure-\ref{DC-Fig} depicts how this would work\cite{Eiben95}:
\begin{figure}[h!]
\centering
\begin{tabular}{ | c | c | c | c | c | }
\hline
$a_1$ & $a_2$ & $a_3$ & $a_4$ & $\text{ parent a }$ 	\\ \hline
$b_1$ & $b_2$ & $b_3$ & $b_4$ & $\text{ parent b }$ 	\\ \hline
$c_1$ & $c_2$ & $c_3$ & $c_4$ & $\text{ parent c }$ 	\\ \hline
$d_1$ & $d_2$ & $d_3$ & $d_4$ & $\text{ parent d }$ 	\\ \hline
\end{tabular}
$\rightarrow$
\begin{tabular}{ | c | c | c | c | c | }
\hline
$a_1$ & $d_2$ & $c_3$ & $b_4$ & $\text{ child a }$ 	\\ \hline
$b_1$ & $a_2$ & $d_3$ & $c_4$ & $\text{ child b }$ 	\\ \hline
$c_1$ & $b_2$ & $a_3$ & $d_4$ & $\text{ child c }$ 	\\ \hline
$d_1$ & $c_2$ & $b_3$ & $a_4$ & $\text{ child d }$ 	\\ \hline
\end{tabular}
\caption{Diagonal crossover applied to four parents}
\label{DC-Fig}
\end{figure} 

The rationale behind expanding this operator was to increase the disruptiveness, and by extension the explorativity, of Genetic Algorithms\cite{Eiben97}. This meant that the population would have to have a large degree of similar genetic sequences before the search would narrow and converge\cite{Eiben95}. It should also be noted that in the special case of $n = 2$ is identical to traditional 1-point crossover\cite{Eiben95}.