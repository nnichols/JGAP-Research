Genetic Algorithm research borrows terms from both Biology and Computer Science, so we will set up some conventions on terminology for this paper. A \emph{genetic sequence}, often also called an \emph{individual}, is an encoding of a potential solution to a particular problem. Each value within a genetic sequence is referred to as an \emph{allele}. A \emph{schema} is a genetic sequence where a subset of the alleles are allowed to have their value left unspecified \cite{Russell10}. A \emph{genetic operator} of \emph{arity} $n$ is a function that modifies $n$ \emph{parent} genetic sequences to produce $m$ new, \emph{child}, genetic sequences. Historically, the phrase \emph{mutation operator} has referred to the case where $n = 1$, \emph{recombination operator} has specified $n = 2$, and \emph{multi-parent recombination operator} was reserved for $n > 2$ \cite{Eiben94}. A \emph{population} is the set of every individual for a given instance of a Genetic Algorithm. Finally, a \emph{Memetic Algorithm} is a Genetic Algorithm where each individual undergoes a quick local search as an additional genetic operator of arity 1.