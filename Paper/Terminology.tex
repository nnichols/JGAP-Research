Genetic Algorithm research borrows terms from both Biology and Computer Science, so we have set several conventions on terminology for this paper. A \emph{Genetic Algorithm} is a local search technique that simulates biological evolution on solutions to a given problem\cite{Russell10}. A \emph{genetic sequence} $g_k$, often also called an \emph{individual}, is a sequence of values $a_1 a_2\ldots a_n$ that represents a potential solution to a particular problem. Each value $a_i$ within a genetic sequence is referred to as an \emph{allele} and can take on values from a specified domain $D_i$, commonly the set $\{0,1\}$. A \emph{population} refers to the set $G$ whose elements are the genetic sequences for a given instance. A \emph{schema}, pluralized as \emph{schemata}, is a partial genetic sequence where each $a_i$ can be left unspecified and is not required to take any value \cite{Russell10}. In this paper, the character `-' will represent a value that has been left unspecified, which is commonly called a \emph{don't care} value \cite{Holland75}.

Genetic Algorithms have two distinct phases, \emph{recombination} and \emph{selection}, which occur every iteration, also called a \emph{generation}. During the selection phase, a \emph{fitness function} takes a $g_k \in G$ as input and returns an $x \in \mathbb{R}$. This $x$, or \emph{fitness value}, measures how well $g_k$ solves a given problem instance. \emph{Elitism} is a strategy that takes the individual with the highest fitness value in a population and copies it into the next generation.

Recombination utilizes \emph{genetic operators}, which are functions from genetic sequences to genetic sequences. The probability of a particular genetic sequence being chosen for selection is typically determined by its fitness value. Genetic operators of \emph{arity} $n$ take $n$ \emph{parent} genetic sequences to produce $m$ new, \emph{child}, genetic sequences. It should be noted that $m$ and $n$ are two not necessarily distinct natural numbers.  Historically, the phrase \emph{mutation operator} has referred to the case where $n = 1$, \emph{recombination operator} has specified $n = 2$, and \emph{multi-parent recombination operator} was reserved for $n > 2$ \cite{Eiben94}. The focuses of this paper, \emph{genetic overlays} and \emph{elitist schema overlays}, are defined in the following section.