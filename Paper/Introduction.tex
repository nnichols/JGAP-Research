Potential solutions to optimization and search problems can be generated and have their quality checked in a much shorter amount of time than solving them. Thus, a natural strategy for problems like this would be to generate a large, random pool of solutions and use the quality checks to find better solutions. This is central to how Genetic Algorithms operate. We can encode solutions to these problems as genetic sequences and then replicate Evolution on them. The quality scores let us score a solution's fitness, via the \emph{selection} operator. This information helps us choose which solutions we want to derive additional potential solutions from, which is done by a \emph{recombination} operator. Finally, it is helpful to change small components in these new solutions to expand our search with the \emph{mutation} operator \cite{Deb99}.

By simulating the natural process of evolution we can quickly generate solutions to computationally difficult problems like these. Though they are not guaranteed to find the optimum solution or a goal state, their limited resource consumption and ability to quickly find good approximations are very attractive\cite{Russell10}. Much research has gone into improving these approximations by modifying the strategies and heuristics used, and this paper aims to do the same.

Historically, Genetic Algorithms have been modeled closely after observations from Biology. For example, both biological reproduction and recombination in traditional Genetic Algorithms always occurs between either 1 or 2 parent(s); However, this is only a restriction in Biology\cite{Eiben95}. For Genetic Algorithms, the number of parents used during recombination can easily be expanded beyond 2. Papers describing techniques for multi-parent recombination started appearing as early as 1966 but little was reported about their behavior early on \cite{Eiben03}. The strategies investigated showed promise, and they have since drawn more attention and research\cite{Eiben94}. 

Curiously, many of these new operators were extensions of traditional crossover operators modified only to accommodate for more parents. In this paper, we aim to follow the trend of diverging from the restrictions of Biology by introducing a new methodology, \emph{genetic overlay}. This operator aims to capitalize on the benefits of both crossover and fitness-based scanning. This combination should allow us to exploit problems in which good schemata are closely paired and related to good solutions by trying to identify the genes correlated with high fitness scores \cite{Russell10}. 
