Solving optimization and search problems takes far more time than generating sets of random potential solutions and checking how well they approximate a solution to the problem. Thus, a natural strategy for problems like this would be to generate a large, random pool of solutions and use information from a quality check to find better solutions. This is central to how Genetic Algorithms operate. We can encode solutions to these problems as genetic sequences and then replicate the process of natural evolution on them. The quality checks let us score a solution's fitness, via the \emph{selection} operator. This information helps us choose which solutions we want to derive additional potential solutions from, which is done by a \emph{recombination} operator. Finally, it is helpful to change small components in these new solutions to expand our search with a \emph{mutation} operator \cite{Deb99}.

This methodology lets us quickly generate approximate solutions to these problems. Though Genetic Algorithms are not guaranteed to find the optimum solution or a goal state, their limited resource consumption and ability to quickly find good approximations is very attractive\cite{Russell10}. Much research has gone into improving these approximations by modifying the strategies and heuristics used, and this paper aims to do the same.

Historically, Genetic Algorithms have been modeled closely after observations from Biology. For example, both biological reproduction and recombination in traditional Genetic Algorithms always occurs between either 1 or 2 parent(s); however, this is only a restriction in Biology\cite{Eiben95}. For Genetic Algorithms, the number of parents used during recombination can easily be expanded beyond 2. Papers describing techniques for multi-parent recombination started appearing as early as 1966 but little was reported about their behavior early on \cite{Eiben03}. The strategies investigated showed promise, and they have since drawn more attention and research\cite{Eiben94}. 

Curiously, many of these new operators were extensions of traditional recombination operators modified only to accommodate for more parents. In this paper, we follow the trend of diverging from the restrictions of Biology by introducing a new genetic operator, the \emph{elitist schema overlay}, and the more general concept of a \emph{genetic overlay}. This operator combines the benefits of both crossover and fitness-based scanning by attempting to identify and propogate the genes correlated with the most successful solutions that have been discovered \cite{Russell10}. 