The selection phase of Genetic Algorithms determines the probability of a given individual being chosen for recombination based upon fitness. This is necessary in order to preserve the genes that performed well. This was conceptually expanded upon with elitism, allowing the best performing individual to survive into the next generation. Elitist schema overlays engrain this concept within a genetic operator, and are a special instance of genetic overlays:

\begin{overlay}
Genetic Overlay - A genetic operator which modifies an individual's genetic sequence to match the specified alleles of a given schema, while leaving the unspecified alleles at their original values. 
\end{overlay}

\begin{figure}[h!]
\centering 
\begin{align*}
\text{Initial Individual} &: 0~1~0~0~1~0 		\\
\text{Genetic Overlay} &: 1-0~1-0				\\
\text{Resultant Individual} &: 1~1~0~1~1~0			
\end{align*}

\caption{Genetic overlay on a binary genetic sequence}
\label{GO-Fig}
\end{figure}

Figure-\ref{GO-Fig} has simple example of a binary genetic overlay where the unspecified alleles have been marked with a dash. The \emph{defining length} of a schema is the number of specified values, in this case $4$.

This operator allows us to quickly give a set of individuals very similar genetic sequence, thus allowing us to search the neighborhood of the provided schema. The choice and density of the schema that we use is very important. If our schema is very dense, i.e. it consists of mostly specified alleles, then the population will converge very quickly. Early convergence is problematic because it lessens the probability that the optimal solution has been found\cite{Deb99}. Likewise, a very sparse schema will spend excess computation cycles applying very few changes. Additionally, we want to ensure that our schema consists of alleles that yield high fitness values. Each of these factors must be carefully considered when developing a strategy for creating genetic overlays. These will be investigated further in the following section.

%
% Holland Quote - pg. 66
% Goldberg Information - pg. 41
%
A close relationship can be seen between this operator, the schema theorem, and the building-block hypothesis\cite{Goldberg89, Holland75}. The schema theorem, in Holland's own words, states, "The adaptive system must, as an integral part of its search of $a$, persistently test and incorporate structural properties associated with better performance \cite{Holland75}." Thus, by identifying the alleles associated with high performance, we can use a genetic overlay to incorporate them into our entire population. This will lead to the schemata being tested by the fitness function more often since it is present in more individuals. The building block hypothesis states that the juxtaposition of building blocks, or schemata that are have high fitness values and low defining lengths, are integral to the successes of Genetic Algorithms \cite{Goldberg89}. Therefore, we should be careful to build genetic overlays that are correlated to high fitness values while leaving ample room to search within.

Strangely, when Forrest and Mitchell tested the performance of Genetic Algorithms against Hill-Climbing Algorithms on the Royal Road function, whose definition is tightly coupled with the building-block hypothesis, Genetic Algorithms were out-performed\cite{Forrest93} by Random Hill-Climbing Algorithms. Results like this have lead to criticism of the strength of the underlying assumptions and the narrowness over which the schema theorem and building-block hypothesis could be applied\cite{Burjorjee08, Senaratna05}. This has lead to the use of effective fitness measures and coarser graining on the size of building-block schemata to describe the evolutions of schemata over time\cite{Stephens99}.

The underlying mechanics of successful Genetic Algorithms are still being debated, but the exploitation of current, successful genetic sequences is still fundamental to the field as a whole\cite{Russell10, Senaratna05}. Genetic overlays were designed to speed up the propagation of schemata in a population, and elitist schema overlays search the fittest individuals for useful schemas to propogate.