The selection phase of Genetic Algorithms determines the probability of a given individual being chosen for recombination based upon fitness. This is necessary in order to preserve the genes that performed well. This was conceptually expanded upon with elitism, allowing the best performing individual to survive into the next generation. Elitist schema overlays engrain this concept within a genetic operator, and are a special instance of genetic overlays:

\begin{overlay}
Genetic Overlay - A genetic operator which modifies an individual's genetic sequence to match the specified alleles of a given schema, while leaving the unspecified alleles at their original values. 
\end{overlay}

Figure-\ref{GO-Fig} has an example of a genetic overlay that operates upon binary genetic sequences. The \emph{defining length} of a schema is defined to be the number of specified values in an overlay, in the case of the provided example, we have the value $4$.

\begin{figure}[h!]
\centering 
\begin{align*}
\text{Initial Individual} &: 0~1~0~0~1~0 		\\
\text{Genetic Overlay} &: 1-0~1-0				\\
\text{Resultant Individual} &: 1~1~0~1~1~0			
\end{align*}

\caption{Genetic Overlay Example}
\label{GO-Fig}
\end{figure}


This operator allows us to quickly give a set of individuals very similar genetic sequences, thus allowing us to search the neighborhood of the provided schema. The choice and density of the schema that we use is very important. If our schema is very dense, i.e. it consists of mostly specified alleles, then the population will converge very quickly. Early convergence is problematic because it lessens the probability that the optimal solution has been found\cite{Deb99}. Likewise, a very sparse schema will spend excess computation cycles applying very few changes. Additionally, we want to ensure that our schema consists of alleles that yield high fitness values. Each of these factors must be carefully considered when developing a strategy for creating genetic overlays. 

\subsection*{Elitist Schema Overlays}
There are many reasonable methodologies that we could utilize to build effective genetic overlays. The central focus of this paper is the use of \emph{elitist schema overlays}, and they are conceptually defined below.

\begin{overlay}
Elitist Schema Overlay - A genetic overlay that is built from the matching alleles of the elite sub-population.
\end{overlay}

To define this formally, let $P = \{p_1,p_2,\ldots,p_n\}$ be a set of $n$ individuals selected for recombination. Consider $T=\{t_1,t_2,\ldots,t_k\}$ as the set of the $k$ individuals with the highest fitness rankings in the population. We will now construct the genetic overlay, named $s$, from the genetic sequences in $T$ to be applied to each $p_j \in P$. Let $C = \{c_1,c_2,\ldots,c_n\}$, be the $n$ genetic sequences resulting from this application. The notations $s[i], t[i], c_j[i], \text{ and } p_j[i]$ represent the value, or lack-thereof in the case of the schema, at position $i$. 
 
 \begin{displaymath}
   s[i] = \left\{
     \begin{array}{cl}
       t_1[i] & \text{if~ } t_1[i] = t_2[i]) = \ldots = t_k[i] \\
       - & \text{if~ } \text{otherwise} 
     \end{array}
   \right.
\end{displaymath} 

Figure-\ref{ESO-Fig} demonstrates what this process looks like for $k = 3$.

\begin{figure}[h!]
\centering 
\begin{align*}
\text{Elite Individual 1} &: 0~1~0~0~1~0~1~1~0 			\\
\text{Elite Individual 2} &: 0~1~0~0~0~0~1~0~0 			\\
\text{Elite Individual 3} &: 1~1~0~1~1~0~1~1~0 			\\  
\text{Resultant Genetic Overlay} &:   -1~0--0~1-0			
\end{align*}
\caption{Production of an Elitist Schema Overlay}
\label{ESO-Fig}
\end{figure}

Thus, our schema will only have specified values where every individual in $T$ has the same value at that same position. From here, we will apply $s$ as a genetic overlay to each individual in $P$ to define the genetic sequences in $C$. The value of $c_j[i]$ is defined below:

 \begin{displaymath}
   c_j[i] = \left\{
     \begin{array}{cl}
       s[i] & \text{if~ } s[i] \not= - \\
       p_j[i] & \text{if~ } \text{otherwise} 
     \end{array}
   \right.
\end{displaymath} 

This is further explained in figure-\ref{ESO-apl}, which implements the elitist schema overlay created in figure-\ref{ESO-Fig}.

\begin{figure}[h!]
\centering 
\begin{align*}
\text{Elitist Schema Overlay} &:   -1~0--0~1-0			\\  
\text{Unmodified Individual} &: 1~0~0~1~1~0~1~1~1	\\		
\text{Resultant Individual} &: 1~1~0~1~1~0~1~1~0
\end{align*}
\caption{Application of an Elitist Schema Overlay}
\label{ESO-apl}
\end{figure}

At this point, we may continue to apply genetic operators to $P$ or $C$ as need be, or continue to the next generation. If our best performing individuals all have the same values at various alleles, this may hint that these assignments are correlated with success. Thus, it would be reasonable to search the neighborhood of this partial solution more thoroughly. 

Figure-\ref{ESO-Fig} demonstrates how elitist schema overlays are constructed, and figure-\ref{ESO-apl} demonstrates the result of its application. It should be noted that the members of the set $P$ can be selected from the unmodified population or from the population resulting from the application of any other genetic operator. This was done intentionally to explore the behavior of elitist schema overlays as both a mutative and reproductive genetic operator. 

Admittedly, the effects of this method will depend greatly upon our choice of $k$. For instance, the choice $k = 1$ will replace every individual that has the genetic overlay applied with the highest ranked member of the population. This is problematic, since it will instantly lead us to convergence. Likewise, a large value of $k$ will greatly reduce the probability that $t_1[i] = t_2[i] = \ldots = t_k[i]$ is true. This would lead us to checking a large number of alleles and, in the likeliest case, doing little to nothing with that information.

To determine reasonable values for $k$, it is helpful to know what the probability of an allele being specified in an elitist schema overlay is. We will denote $\alpha_i$ to be this probability, and $x_i$ to represent the size of the domain of the $i$th allele for a genetic sequence of length $n$. Note that this assumes that each value for an allele has an equal probability of showing up in a selected genetic sequence at allele $i$. If this is not the case, then this gives us a lower bound of the probability that this property is true.

\[ \alpha_i = P(\text{\emph{k} parents matching \emph{i}th allele}) = \left(\frac{1}{x_i}\right)^{k-1} \]

With this, we can easily derive the probabilities of having no matches and a total match for a chromosome of length $n$ in terms of $\alpha_i$.

\begin{align*}
P( \text{No matches})&= \prod\limits_{i = 1}^n (1 - \alpha_i) \\
P( \text{All matches})&= \prod\limits_{i = 1}^n (\alpha_i)
\end{align*}

Thus, our choice of $k$ should be made carefully, but there are a large number of possibilities that we could easily consider. A fixed and predetermined $k$ could be chosen, $k$ could be randomized for each generation, or $k$ could be set to the number of individuals above a certain fitness threshold; however, the probabilities mentioned above lent us to consider an alternative approach. If we create an inverse relationship between the generations elapsed and our choice of $k$ we get some interesting properties:

\begin{enumerate}

\item Narrowing Search - Since the initial $k$ values will be very large, it is unlikely that $t_1(i) = t_2(i) = \ldots = t_k(i)$, and so we will not disturb the initial natural diversity with the operator.  Likewise, as our $k$ drops, our genetic overlay has a higher chance to be mostly specified, meaning that we will be searching a progressively narrowing neighborhood centered about the best performing solutions discovered so far \cite{Neri11}.

\item Bounded Convergence - We can control how quickly $k$ decreases, and by this, how quickly our population will converge towards the best known solutions. When we set $k=1$ we will instantly converge the entire target population of this operator, and so the minimum convergence rate can now be set by the user.

\end{enumerate}

%
% GA Theory
%
% Holland Quote - pg. 66
% Goldberg Information - pg. 41
%
A close relationship can be seen between genetic overlays, the schema theorem, and the building-block hypothesis\cite{Goldberg89, Holland75}. The schema theorem, in Holland's own words, states, "The adaptive system must, as an integral part of its search of $a$, persistently test and incorporate structural properties associated with better performance \cite{Holland75}." Thus, by identifying the alleles associated with high performance, we can use a genetic overlay to incorporate them into our entire population. This will lead to the schemata being tested by the fitness function more often since it is present in more individuals. 

The building block hypothesis states that the propagation of building blocks, high fitness schemata with low defining lengths, are integral to the successes of Genetic Algorithms \cite{Goldberg89}. Therefore, we should be careful to build genetic overlays that are correlated with high fitness values while leaving ample room to search.

Strangely, when Forrest and Mitchell tested the performance of Genetic Algorithms against Hill-Climbing Algorithms on the Royal Road function, whose definition is tightly coupled with the building-block hypothesis, Genetic Algorithms were out-performed by Random Hill-Climbing Algorithms\cite{Forrest93}. Results like this have lead to criticism of the strength of the underlying assumptions and the narrowness over which the schema theorem and building-block hypothesis could be applied\cite{Burjorjee08, Senaratna05}. This has lead to the use of effective fitness measures and coarser graining on the size of building-block schemata to describe the evolutions of schemata over time\cite{Stephens99}.

The underlying mechanics of successful Genetic Algorithms are still being debated, but the exploitation of current, successful genetic sequences is still fundamental to the field as a whole\cite{Russell10, Senaratna05}. Genetic overlays were designed to speed up the propagation of schemata in a population, and elitist schema overlays search the fittest individuals for useful schemata to propogate. To make comparative statements about the effectiveness of elitist schema overlays, we have also implemented successful genetic operators from the literature.