Fitness values are the determining force behind whether or not the genetic sequence of an individual will survive and propagate. The selection phase of Genetic Algorithms determines the probability of a given individual being chosen for recombination. The technique to ensure the survival of good genes was conceptually expanded upon with elitism, which allows the best performing individual to survive into the next generation. Elitist schema overlays ingrain this concept within a genetic operator, and are a special instance of genetic overlays:

\begin{overlay}
Genetic Overlay - A genetic operator which modifies an individual's genetic sequence to match the specified alleles of a given schema, while leaving the unspecified alleles at their original values. 
\end{overlay}

The following is an example of a genetic overlay that operates upon binary genetic sequences. The \emph{defining length} of a schema is defined to be the number of specified values in an overlay, in the case of the provided example, we have the value $4$.

\begin{align*}
\text{Initial Individual} &: {\color{myred}0~1~0~0~1~0} 		\\
\text{Genetic Overlay} &: {\color{myblue}1-0~1-0}				\\
\text{Resultant Individual} &: {\color{myblue}1}~{\color{myred}1}~{\color{myblue}0~1}~{\color{myred}1}~{\color{myblue}0}	
\end{align*}

This operator allows us to quickly give a set of individuals very similar genetic sequences, thus allowing us to search the neighborhood of the provided schema. The choice and density of the schema that we use is very important. If our schema is very dense, i.e. it consists of mostly specified alleles, then the population will converge very quickly. Early convergence is problematic because it lessens the probability that the optimal solution has been found \cite{Deb99}. Likewise, a very sparse schema will spend excess computation cycles applying very few changes. Additionally, we want to ensure that our schema consists of alleles that yield high fitness values. Each of these factors must be carefully considered when developing a strategy for creating genetic overlays. 

\subsection*{Elitist Schema Overlays}
There are many reasonable methodologies that we could utilize to build effective genetic overlays. The central focus of this paper is the use of \emph{elitist schema overlays}, and they are defined below.

\begin{overlay}
Elitist Schema Overlay - A genetic overlay that is built from the matching alleles of the elite sub-population.
\end{overlay}

To define this formally, let $P = \{p_1,p_2,\ldots,p_n\}$ be a set of $n$ individuals selected for recombination. Consider $T=\{t_1,t_2,\ldots,t_k\}$ as the set of the $k$ individuals with the highest fitness rankings in $P$. We will now construct the genetic overlay, named $s$, from the genetic sequences in $T$. We want our genetic overlay to  have specified values only where every individual in $T$ has the same value at that same position. The notations $s[i], t[i]_j, \text{ and } p[i]$ represent the value, or lack-thereof in the case of the schema, at position $i$. 
 
 \begin{displaymath}
   s[i] = \left\{
     \begin{array}{cl}
       t_1[i] & \text{if~ } t_1[i] = t_2[i] = \ldots = t_k[i] \\
       - & \text{if~ } \text{otherwise} 
     \end{array}
   \right.
\end{displaymath} 

\noindent Below is an example for $k = 3$.

\begin{align*}
\text{Elite Individual 1} &: 0~1~0~0~1~0~1~1~0 			\\
\text{Elite Individual 2} &: 0~1~0~0~0~0~1~0~0 			\\
\text{Elite Individual 3} &: 1~1~0~1~1~0~1~1~0 			\\  
\text{Resultant Genetic Overlay} &:   -1~0--0~1-0			
\end{align*}

From here, we will apply $s$ as a genetic overlay to each individual in $P$. An application of the example elitist schema overlay created before can be seen below:

\begin{align*}
\text{Unmodified Individual} &: {\color{myred}1~0~0~1~1~0~1~1~1}		\\	
\text{Elitist Schema Overlay} &:   {\color{myblue}-1~0--0~1-0	}		\\  	
\text{Resultant Individual} &:  {\color{myred}1}~{\color{myblue}1~0}~ {\color{myred}1~1}~{\color{myblue}0~1}~ {\color{myred}1}~{\color{myblue}0}
\end{align*}


At this point, we may continue to apply additional genetic operators to $P$ as need be, or continue to the next generation. If our best performing individuals all have the same values at various alleles, this may hint that these assignments are correlated with success. Thus, it would be reasonable to search the neighborhood of this partial solution more thoroughly. 

It should be noted that the members of the set $P$ can be selected from the unmodified population or from the population resulting from the application of any other genetic operator. This was done intentionally to explore the behavior of elitist schema overlays as both a mutative and reproductive genetic operator. 

Admittedly, the effects of this method will depend greatly upon our choice of $k$. For instance, the choice $k = 1$ will replace every individual that has the genetic overlay applied with the highest ranked member of the population. This is problematic, since it will instantly lead us to convergence. Likewise, a large value of $k$ will greatly reduce the probability that $t_1[i] = t_2[i] = \ldots = t_k[i]$ is true. This would lead us to checking a large number of alleles and, in the likeliest case, doing little to nothing with that information.

To determine reasonable values for $k$, it is helpful to know what the probability of an allele being specified in an elitist schema overlay is. We will denote $\alpha_i$ to be this probability, and $x_i$ to represent the size of the domain of the $i$th allele for a genetic sequence of length $n$. Note that this assumes that each value for an allele has an equal probability of showing up in a selected genetic sequence at allele $i$. Given that this assumption is rarely true, as more successful alleles should be present with a higher frequency, the following is a lower bound of the probability of a matching allele over $k$ parents.

\[ \alpha_i = P(\text{\emph{k} parents matching \emph{i}th allele}) = \left(\frac{1}{x_i}\right)^{k-1} \]

With this, we can easily derive the probabilities of having no matches and a total match for a chromosome of length $n$ in terms of $\alpha_i$.

\begin{align*}
P( \text{No matches})&= \prod\limits_{i = 1}^n (1 - \alpha_i) \\
P( \text{All matches})&= \prod\limits_{i = 1}^n (\alpha_i)
\end{align*}

Thus, our choice of $k$ should be made carefully, but there are a large number of possibilities that we could easily consider. Large values for $k$ will make $\alpha_i$ small, and thus the probability that no matches have been found will be high. Likewise, low values for $k$ increase the value of $\alpha_i$, making the probability of every allele matching high. We need to find a methodology for selecting $k$ that balances these two extremes. Else, we run the risk of wasting computations looking for very unlikely matches or causing our population to converge prematurely. A fixed and predetermined $k$ could be chosen, $k$ could be randomized for each generation, or $k$ could be set to the number of individuals above a certain fitness threshold; however, the probabilities mentioned above lead to an alternative approach. We can create an inverse linear relationship between $k$ and the number of generations remaining. By changing the value of $k$ over time, we hope to achieve the following properties:

\begin{enumerate}

\item Narrowing Search - Since the initial $k$ values will be very large, it is unlikely that $t_1(i) = t_2(i) = \ldots = t_k(i)$, and so we will not disturb the initial natural diversity with the operator.  Likewise, as our $k$ drops, our genetic overlay has a higher chance to be mostly specified, meaning that we will probably be searching a progressively narrowing neighborhood based upon the best performing solutions discovered so far \cite{Neri11}.

\item Bounded Convergence - We can control how quickly $k$ decreases, and by this, how quickly our population will converge towards the best known solutions. When we set $k=1$ we will instantly converge the entire target population of this operator, and so we have a bound for the minimum convergence rate.

\end{enumerate}

Our tests compare how well elitist schema overlays perform with a random $k$ being selected each generation, $k$ being set as a fixed percent of the population, and $k$ being set by a linear inverse relationship with the number of generations remaining.

%
% GA Theory
%
% Holland Quote - pg. 66
% Goldberg Information - pg. 41
%
\subsection*{Rationale}
Elitist schema overlays with an effective choice of $k$, are intentionally related to the theoretical bases of Genetic Algorithms \cite{Goldberg89, Holland75}. The schema theorem, in Holland's own words, states, ``The adaptive system must, as an integral part of its search of $a$, persistently test and incorporate structural properties associated with better performance \cite{Holland75}.'' Likewise, the building block hypothesis states that the propagation of building blocks, high fitness schemata with low defining lengths, are integral to the successes of Genetic Algorithms \cite{Goldberg89}. Thus, if we can identify the alleles associated with high performance, then we can use a genetic overlay to incorporate them into our entire population. This will lead to the identified schemata being tested by the fitness function more often since it is present in more individuals. Since premature convergence will prevent solutions from improving much more, we should be careful to only build genetic overlays that are correlated with high fitness values while leaving ample room to search \cite{Andre01}.

Strangely, when Forrest and Mitchell tested the performance of Genetic Algorithms against Hill-Climbing Algorithms on the Royal Road function, whose definition is tightly coupled with the building-block hypothesis, Genetic Algorithms were out-performed by Random Hill-Climbing Algorithms \cite{Forrest93}. Results like this have lead to criticism of the strength of the underlying assumptions and the narrowness over which the schema theorem and building-block hypothesis could be applied \cite{Burjorjee08, Senaratna05}. This has lead to the use of effective fitness measures and coarser graining on the size of building-block schemata to describe the evolutions of schemata over time \cite{Stephens99}.

The underlying mechanics of successful Genetic Algorithms are still being debated, but the exploitation of current, successful genetic sequences is still fundamental to the field as a whole \cite{Russell10, Senaratna05}. Genetic overlays were designed to speed up the propagation of schemata in a population, and elitist schema overlays search the fittest individuals for useful schemata to propagate. This operator will be explored as an addition to current models, modifying the population between the recombinant operators, fitness-based scanning and diagonal crossover, and the mutation operator. 

This decision was made because elitist schema overlays act similar to both recombinant and mutative operators. Since they are produced from $k$ genetic sequences, we can consider them as multi-parent operators; however, the application of a genetic overlay is a unary procedure. Thus, we utilize elitist schema overlays as an addition to, not a replacement for, current genetic operators. To determine the effects of including elitist schema overlays in Genetic Algorithms, we will now define the test problems used for our experimentation.