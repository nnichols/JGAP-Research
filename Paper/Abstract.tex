Genetic Algorithms are programs inspired by natural evolution used to solve difficult problems in Mathematics and Computer Science. The theoretical foundations of Genetic Algorithms, the schema theorem and the building-block hypothesis, state that the success of Genetic Algorithms stems from the propagation of fit genetic subsequences. Multi-parent operators were shown to increase the performance of Genetic Algorithms by increasing the disruptivity of genetic operations. Disruptive genetic operators help prevent suboptimal genetic sequences from propagating into future generations, which leads to an improved fitness for the population over time. In this paper we explore the use of a novel multi-parent genetic operator, the \emph{elitist schema overlay}, which propagates the matching segments in the genetic sequences of the elite subpopulation to bias the global search towards the best known solutions. We investigate the parameters that drive the behavior of elitist schema overlays to determine the most successful model, and we compare this to successful multi-parent and traditional genetic operators from the literature. 