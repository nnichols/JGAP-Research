In this paper, we have introduced elitist schema overlays and their conceptual basis, genetic overlays. We tested this new genetic operator against two multi-parent genetic operators from the literature, diagonal crossover and fitness-based scanning \cite{Eiben94}. We have also utilized the multi-parent operators in multi-operator configurations, a concept that has also been supported by the literature \cite{Smith04}.

Our data supports the hypothesis that multi-parent strategies are an improvement upon traditional Genetic Algorithms. When paired with traditional Genetic Algorithms, elitist schema overlays improved performance in most cases, especially while testing instances of the Traveling Salesman Problem. Traditional Genetic Algorithms were also out performed by the operators from the literature: $n$-arity diagonalization, fitness-based scanning, and multi-operator strategies. 

While elitist schema overlays increased the performance of traditional Genetic Algorithms, this effect did not carry over when elitist schema overlays were used with multi-parent operators. In most of the observed cases, adding elitist schema overlays to other $n$-arity genetic operators degraded performance; however, they still outperformed traditional Genetic Algorithms in most cases. While elitist schema overlays could not match the successes of other multi-parent operators, they provide another method to improve upon traditional paradigms.

\subsection*{Further Research}
During our research, we came across and developed several unanswered questions that could be the focus of further research into elitist schema overlays and multi-parent Genetic Algorithms. 

\begin{itemize}
\item What do these results tell us about the building block hypothesis and the schema theorem?

\item What other ways of choosing and modifying the $k$ value in an elitist schema overlay will produce better behavior, and why?

\item What other strategies exist for creating effective genetic overlays?

\item Why do the different percent values for $k$ perform so differently from each other, especially in terms of convergence?

\item Why do elitist schema overlays only produce gains when used in conjunction with traditional Genetic Algorithms, and degrade performance when used with other multi-parent genetic operators?
\end{itemize}